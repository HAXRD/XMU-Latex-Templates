%%%%%%%%%%%%%%%%%%%%%%%%%%%%%%%%
%            宏 包              %
%%%%%%%%%%%%%%%%%%%%%%%%%%%%%%%%
%=== 字体支持 ===
\documentclass{ctexart}             % 中文纸张类
\usepackage{ctex}                   % 中文字体宏包
%=== 数学支持 ===
\usepackage{amsmath}                % AMS宏包
%=== 图形支持 ===
\usepackage{graphicx}
\usepackage{caption}
\usepackage{subcaption}
%=== 超链接支持 ===
\usepackage{hyperref}               % 文本内超链接引用
%=== 引 用 ===
\usepackage{biblatex}
\addbibresource{ref.bib}

%=== 页面排版 ===
\usepackage{geometry}               % 页边距调整
\geometry{top=3.0cm,bottom=2.7cm,left=2.7cm,right=2.7cm}
\usepackage{fancyhdr}               % 页头与页脚
\pagestyle{fancy}
\fancyhf{}
\fancyhead[L]{\zihao{5} \kaishu 主标题}
\fancyfoot[C]{~\zihao{5} \thepage~}
\renewcommand{\headrulewidth}{.65pt}
\usepackage{setspace}               % 行距等设置
%=== 按章编号 ===
\numberwithin{table}{section}
\numberwithin{figure}{section}
%=== 自定义指令 ===
\newcommand{\HRule}{\rule{\linewidth}{0.5mm}}           % 排版线

%=== 章节重命名 ===
\CTEXsetup[format={\centering\bfseries\zihao{-2}},name={第, 章}]{section}
\CTEXsetup[nameformat={\bfseries\zihao{3}}]{subsection}
\CTEXsetup[nameformat={\bfseries\zihao{4}}]{subsubsection}


%=== 乱码生成器 ===
\usepackage{zhlipsum}               % 中文乱码生成器
\usepackage{lipsum}                 % 英文乱码生成器


%===============

%===============



%%%%%%%%%%%%%%%%%%%%%%%%%%%%%%%%
%            正 文              %
%%%%%%%%%%%%%%%%%%%%%%%%%%%%%%%%
\begin{document}
%=== 封 面 ===
\begin{titlepage}
    \begin{center}
        %--- 图 标 ---
        \includegraphics[width=0.4\textwidth]{imgs/r6-logo.png}\\[1cm]
        %--- 标 题 ---
        \textsc{\LARGE \bfseries 部门信息}\\[1.5cm]
        \textsc{\Large 副标题}\\[0.5cm]
        \HRule \\[0.4cm]
        {\huge \bfseries 主标题}\\[0.4cm]
        \HRule \\[1.5cm]
        %--- 作 者 ---
        \begin{minipage}{0.4\textwidth}
            \begin{center} \large
                \textsc{\kaishu 作者1~?}\\
                \textsc{\kaishu 作者2~?}\\
                \textsc{\kaishu 作者3~?}\\
                \textsc{\small\kaishu (按字母序排序)}
            \end{center}
        \end{minipage}
        \vfill
        %--- 日 期 ---
        {\large \today}
    \end{center}
\end{titlepage}

%=== 摘 要 ===
\doublespacing
\begin{abstract}
    \zhlipsum
\end{abstract}

%=== 目 录 ===
\clearpage
\tableofcontents
\clearpage

%=== 章 节 ===
\section{Section}
\zhlipsum[2]
\subsection{示例}
\zhlipsum[3]
\subsubsection{插入图形}
\zhlipsum[1]
\begin{figure}
    \centering
    \begin{subfigure}[b]{0.2\textwidth}
        \centering
        \includegraphics[width=\textwidth]{imgs/SAS-Mute.png}
        \caption{Mute}
        \label{fig:mute}
    \end{subfigure}
    \hfill
    \begin{subfigure}[b]{0.2\textwidth}
        \centering
        \includegraphics[width=\textwidth]{imgs/SAS-Sledge.png}
        \caption{Sledge}
        \label{fig:sledge}
    \end{subfigure}
    \hfill
    \begin{subfigure}[b]{0.2\textwidth}
        \centering
        \includegraphics[width=\textwidth]{imgs/SAS-Smoke.png}
        \caption{Smoke}
        \label{fig:smoke}
    \end{subfigure}
    \hfill
\begin{subfigure}[b]{0.2\textwidth}
        \centering
        \includegraphics[width=\textwidth]{imgs/SAS-Thatcher.png}
        \caption{Thatcher}
        \label{fig:thatcher}
    \end{subfigure}
    \caption{S.A.S Operators}
    \label{}
\end{figure}

\subsubsection{引用示例}
爱因斯坦写了相对论\cite{einstein}。

\begin{table}[]
    \begin{tabular}{|l|l|}
    \hline
    \multicolumn{1}{|c|}{函数名称}                           & \multicolumn{1}{c|}{功能}                                                        \\ \hline
    \textit{uni.openBluetoothAdapter(OBJECT)}            & 初始化蓝牙模块                                                                        \\ \hline
    \textit{uni.startBluetoothDevicesDiscovery(OBJECT)}  & 开始搜索附近的蓝牙外围设备                                                                  \\ \hline
    \textit{uni.onBluetoothDeviceFound(CALLBACK)}        & 监听寻找到新设备的事件                                                                    \\ \hline
    \textit{uni.onBluetoothAdapterStateChange(CALLBACK)} & 监听蓝牙适配器状态变化                                                                    \\ \hline
    \textit{uni.closeBluetoothAdapter(OBJECT)}           & 关闭蓝牙模块                                                                         \\ \hline
    \textit{uni.writeBLECharacteristicValue(OBJECT)}     & 向低功耗蓝牙设备特征值中写入二进制数据                                                            \\ \hline
    \textit{uni.createBLEConnection(OBJECT)}             & 连接低功耗蓝牙设备                                                                      \\ \hline
    \textit{uni.onBLEConnectionStateChange(CALLBACK)}    & \begin{tabular}[c]{@{}l@{}}监听低功耗蓝牙连接状态的改变事件(包括\\ 主动连接或断开连接、设备丢失等)\end{tabular} \\ \hline
    \textit{uni.getBLEDeviceServices(OBJECT)}            & 获取蓝牙设备所有服务                                                                     \\ \hline
    \textit{uni.getBLEDeviceCharacteristics(OBJECT)}     & 获取蓝牙设备某个服务中所有特征值                                                               \\ \hline
    \textit{uni.closeBLEConnection(OBJECT)}              & 断开与低功耗蓝牙设备的连接                                                                  \\ \hline
    \end{tabular}
    \caption{表格示例}
\end{table}
%=== 参考文献 ===
\clearpage
\printbibliography
\end{document}